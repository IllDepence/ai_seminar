\section{Introduction}
The classic types of problems dealt with in the field of artificial intelligence planning are problem-solving tasks where the length of plans and the time for finding them are to be minimized. This is referred to as ''classical planning''. AI planning in this traditional sense is already applicable to various practical tasks, such as route planning or logistics. Additional areas of application can be found where the task at hand can be modelled in a way that is structurally similar to classical planning problems. One such area is \emph{story telling}, where AI planning has been applied for well over a decade and currently is the dominant technology in use\cite{Porteous10}.

For a first intuition of how ''story telling by means of planning'' can look like, one can consider a story as being modelled as a sequence of actions performed by story characters. Given an initial world state, a goal and a set of characters with certain actions they can perform, a planner can create a coherent story leading from its beginning to the end. This is, of course, just a very basic notion. The inherent differences between classic problem-solving tasks and stories lead to significant deviations from techniques in classical planning and interesting challenges when more sophisticated approaches to narrative generation are considered.

The remainder of this document is an introduction to story telling as an application area of AI planning. It is structured as follows: section~\ref{differences} ''\nameref{differences}'' 
