\section{Conclusion}\label{conclusion}
This seminar report gave a brief introduction into story telling --- more specifically, narrative generation --- as an application area for AI planning. On a \emph{very} basic level stories can intuitively be described in a way that is structurally similar to classic planning problems. Sophisticated approaches, however, quickly reveal challenges that arise due to the difference of classic planning problems and narratives. In order to deal with those, specialized planners can be used, but it is also possible to include nontrivial aspects of narratives into the modeling of the planning problem such that a classical planner can be applied. Futhermore, stories themselves can be approached on different levels. Story structures (\emph{fabulae}) as well as story presenations (\emph{discourses}) can be generated by means of planning.

Interactive story telling requires temporally bound narrative \mbox{(re-)}generation due to non deterministic intervention. AI planning has been applied in this field for well over a decade and currently is the dominant technology in use \cite{Porteous10}. This shows that planning can be the first choice approach even in areas that are, at first glance, quite different from classic problem-solving tasks. In a recent paper Ware et al. \cite{Ware15} present a video game that uses planning for interactive story telling. This suggests that AI planning could become a key instrument for digital media formats such as games in the future.% maybe sth. about new digital media and the possible shape of things to come in that regard (we'll see continued work in the field of IS because games)

% - seen approach at different levels
% - seen application areas where useful
% - maybe find reference realy world in use application example
