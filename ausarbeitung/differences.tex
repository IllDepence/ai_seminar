\section{Differences to classical planning}\label{differences}
Although the process of generating a story can be modeled as a planning task, the underlying nature of a classic planning problem and a narrative is very different. Both the solution to a planning problem and a narrative are a sequence of successive steps, leading from a beginning to an end. A desired quality of a classic planning problem solution is shortness. Qualities sought in narratives, however, are less easily quantifiable, aesthetic properties such as ''interestingness'' and ''originality''. While there is no obvious solution as to how such properties can be encoded in a planning problem and optimised for, it is easily recognizable that shortness is not an optimum. A basic approach towards generating interesting stories is introducing variation. For a given, rough outline of a story variations in character actions (within reasonable bounds of the story world) or changes in perspective can make for an interesting narrative.

% - nature of a classical problem-solving task
% - nature of (as in what humans look for in) stories
% -> optimal plans not needded, not desired
% -> variation desired

% - in case of IS, replanning
% - conceptual differences (intentionality, etc)
% - potentially further requirements/constraints due to nature of modelling (story knowlege/handling, ...)
