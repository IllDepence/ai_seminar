\section{Differences to classical planning}\label{differences}
Although the process of generating a story can be modeled as a planning task, the underlying nature of a classic planning problem and a narrative is very different. As a consequence there are similarities at a coarse level and differences once more complex properties are taken into account. An essential similarity on a very simple level: both the solution to a planning problem and a narrative are a sequence of successive steps, leading from a beginning to an end. In contrast, a key difference: a desired quality of a classic planning problem solution is shortness. Qualities sought in narratives, however, are less easily quantifiable, aesthetic properties such as ''interestingness'' and ''originality''. While there is no obvious solution as to how such properties can be encoded in a planning problem and optimised for, it is easily recognizable that shortness is not an optimum. A basic approach towards generating interesting stories is introducing variation. For a given, rough outline of a story, variation in character actions (within reasonable bounds of the story world) or changes in perspective can make for an interesting narrative.

Depending on what properties of a narrative are taken into account, further requirements or constraints have to be considered. These can sometimes be incorporated into the modeling of the story world -- such that the resulting planning problem can be solved by an off-the-shelf solver -- some efforts, however, make use of custom a solver. An example for such a solver, tailored for narrative planning, is IPOCL\cite{Riedl04}, which takes causality and character intentionality into account.

Since a common use case for narrative planning is interactive story telling, there is a need for systems capable of replanning in response to changes to the story world introduced by a human element. This is less of a difference between classical and narrative planning in and of themselves but a special requirement for certain narrative planning systemss due to forementioned characteristic use case.
% evtl noch: in case of IS, replanning ✓
%
%
%
% original bullet points:
% - nature of a classical problem-solving task
% - nature of (as in what humans look for in) stories
% -> optimal plans not needded, not desired
% -> variation desired
% - conceptual differences (intentionality, etc)
% - potentially further requirements/constraints due to nature of modelling (story knowlege/handling, ...)
